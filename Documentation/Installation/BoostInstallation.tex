\section{Boost Installation}

\subsection{Linux}
Run your favorite package manager application and install the following packages.

On Ubuntu 10.04
\begin{itemize}
\item
\item 
\item
\end{itemize}

The names of packages and the version may vary depending on the Linux distribution.


\subsection{Windows MinGW}

\subsection{Windows Visual Studio}
The easiest way to install boost on windows is to use an installer. Installers for different versions of Microsoft Visual Studio can be found at
\lstinline{http://www.boostpro.com/products/free}

When installing you will be queried for compiler and variations. Make sure that you select the \emph{Multithread Debug, DLL} and \emph{Multithread, DLL} variations. Boost contains a large number of different libraries. Make sure that you have these installed:
\begin{itemize}
\item \lstinline{file_system}
\item \lstinline{system}
\item \lstinline{thread}
\item \lstinline{program_options}
\end{itemize}

You may install boost whereever you like. We suggest to install it in a folder \lstinline{c:\local\}, to fit the examples in the installation manuals.

\paragraph{Compile and install boost manually}
If you need a version not supported by as binary you can download the source and compile boost yourself. The boost source can be downloaded from www.boost.org. We suggest consulting some of the online tutorials for compiling boost for visual studio. 
